\documentclass[letterpaper,12pt]{article}
\usepackage{array}
\usepackage{threeparttable}
\usepackage{geometry}
\geometry{letterpaper,tmargin=1in,bmargin=1in,lmargin=1.25in,rmargin=1.25in}
\usepackage{fancyhdr,lastpage}
\pagestyle{fancy}
\lhead{}
\chead{}
\rhead{}
\lfoot{}
\cfoot{}
\rfoot{\footnotesize\textsl{Page \thepage\ of \pageref{LastPage}}}
\renewcommand\headrulewidth{0pt}
\renewcommand\footrulewidth{0pt}
\usepackage[format=hang,font=normalsize,labelfont=bf]{caption}
\usepackage{listings}
\lstset{frame=single,
  language=Python,
  showstringspaces=false,
  columns=flexible,
  basicstyle={\small\ttfamily},
  numbers=none,
  breaklines=true,
  breakatwhitespace=true
  tabsize=3
}
\usepackage{amsmath}
\usepackage{amssymb}
\usepackage{amsthm}
\usepackage{harvard}
\usepackage{setspace}
\usepackage{float,color}
\usepackage[pdftex]{graphicx}
\usepackage{hyperref}
\hypersetup{colorlinks,linkcolor=red,urlcolor=blue}
\theoremstyle{definition}
\newtheorem{theorem}{Theorem}
\newtheorem{acknowledgement}[theorem]{Acknowledgement}
\newtheorem{algorithm}[theorem]{Algorithm}
\newtheorem{axiom}[theorem]{Axiom}
\newtheorem{case}[theorem]{Case}
\newtheorem{claim}[theorem]{Claim}
\newtheorem{conclusion}[theorem]{Conclusion}
\newtheorem{condition}[theorem]{Condition}
\newtheorem{conjecture}[theorem]{Conjecture}
\newtheorem{corollary}[theorem]{Corollary}
\newtheorem{criterion}[theorem]{Criterion}
\newtheorem{definition}[theorem]{Definition}
\newtheorem{derivation}{Derivation} % Number derivations on their own
\newtheorem{example}[theorem]{Example}
\newtheorem{exercise}[theorem]{Exercise}
\newtheorem{lemma}[theorem]{Lemma}
\newtheorem{notation}[theorem]{Notation}
\newtheorem{problem}[theorem]{Problem}
\newtheorem{proposition}{Proposition} % Number propositions on their own
\newtheorem{remark}[theorem]{Remark}
\newtheorem{solution}[theorem]{Solution}
\newtheorem{summary}[theorem]{Summary}
%\numberwithin{equation}{section}
\bibliographystyle{aer}
\newcommand\ve{\varepsilon}
\newcommand\boldline{\arrayrulewidth{1pt}\hline}

\begin{document}

\begin{flushleft}
  \textbf{\large{Problem Set \#1}} \\
  MACS 30000, Dr. Evans \\
  Rodrigo Valdes Ortiz
\end{flushleft}

\vspace{5mm}

\noindent\textbf{Data}
Some income data, lognormal distribution and SMM

\textbf{Part (a).} Initial plot.

\begin{figure}[htb]\centering\captionsetup{width=4.0in}
  \caption{\textbf{Histogram of incomes}}\label{FigExample}
  \fbox{\resizebox{4.0in}{3.0in}{\includegraphics{one_ps4.png}}}
\end{figure}

\textbf{Part (b).} Log-normal PDF \newline
Test of the Lognormal PDF with $\mu = 5.0$ and $\sigma = 1.0$ \newline


[[ 0.0019079   0.00123533]
 [ 0.00217547  0.0019646 ]] \newline


\textbf{Part (c).} SMM - 1 step \newline
$\mu_{SMM_1}= 11.3306372028 \quad  \sigma_{SMM_1}= 0.209229380766$ \newline
Data mean of incomes = 85276.8236063 , Data sigma of incomes = 17992.542128 \newline
Model mean 1 = 85276.8254256 , Model sigma 1 = 17992.5434083 \newline
The criterion value is: 5.51827820746e-15 \newline


\begin{figure}[htb]\centering\captionsetup{width=4.0in}
  \caption{\textbf{Histograms of Incomes and SMM 1 step}}\label{FigExample}
  \fbox{\resizebox{4.0in}{3.0in}{\includegraphics{two_ps4.png}}}
\end{figure}


\textbf{Part (d).} SMM - 2 step \newline
$\mu_{SMM_2} = 11.3306371771 \quad \sigma_{SMM_2} = 0.209229372097$ \newline
Data mean of incomes = 85276.8236063 , Data sigma of incomes = 17992.542128 \newline
Model mean = 85276.823074 , Model sigma = 17992.542151 \newline
The criterion value is: 0.0118032258484 \newline


\begin{figure}[htb]\centering\captionsetup{width=4.0in}
  \caption{\textbf{Histograms of Incomes and SMM 1 \& 2 steps}}\label{FigExample}
  \fbox{\resizebox{4.0in}{3.0in}{\includegraphics{three_ps4.png}}}
\end{figure}


\end{document}